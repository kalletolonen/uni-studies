% !TEX TS-program = pdflatex
% !TEX encoding = UTF-8 Unicode

% This is a simple template for a LaTeX document using the "article" class.
% See "book", "report", "letter" for other types of document.

\documentclass[11pt]{article} % use larger type; default would be 10pt

\usepackage[utf8]{inputenc} % set input encoding (not needed with XeLaTeX)

%%% Examples of Article customizations
% These packages are optional, depending whether you want the features they provide.
% See the LaTeX Companion or other references for full information.

%%% PAGE DIMENSIONS
\usepackage{geometry} % to change the page dimensions
\geometry{a4paper} % or letterpaper (US) or a5paper or....
% \geometry{margin=2in} % for example, change the margins to 2 inches all round
% \geometry{landscape} % set up the page for landscape
%   read geometry.pdf for detailed page layout information

\usepackage{graphicx} % support the \includegraphics command and options

% \usepackage[parfill]{parskip} % Activate to begin paragraphs with an empty line rather than an indent

%%% PACKAGES
\usepackage{booktabs} % for much better looking tables
\usepackage{array} % for better arrays (eg matrices) in maths
\usepackage{paralist} % very flexible & customisable lists (eg. enumerate/itemize, etc.)
\usepackage{verbatim} % adds environment for commenting out blocks of text & for better verbatim
\usepackage{subfig} % make it possible to include more than one captioned figure/table in a single float
% These packages are all incorporated in the memoir class to one degree or another...

%%% HEADERS & FOOTERS
\usepackage{fancyhdr} % This should be set AFTER setting up the page geometry
\pagestyle{fancy} % options: empty , plain , fancy
\renewcommand{\headrulewidth}{0pt} % customise the layout...
\lhead{}\chead{}\rhead{}
\lfoot{}\cfoot{\thepage}\rfoot{}

%%% SECTION TITLE APPEARANCE
\usepackage{sectsty}
\allsectionsfont{\sffamily\mdseries\upshape} % (See the fntguide.pdf for font help)
% (This matches ConTeXt defaults)

%%% ToC (table of contents) APPEARANCE
\usepackage[nottoc,notlof,notlot]{tocbibind} % Put the bibliography in the ToC
\usepackage[titles,subfigure]{tocloft} % Alter the style of the Table of Contents
\renewcommand{\cftsecfont}{\rmfamily\mdseries\upshape}
\renewcommand{\cftsecpagefont}{\rmfamily\mdseries\upshape} % No bold!

%%% END Article customizations

%%% The "real" document content comes below...

\title{TIEA3000 Johdatus sulautettuihin järjestelmiin - Harjoitus 1}
\author{Kalle Tolonen}
%\date{} % Activate to display a given date or no date (if empty),
         % otherwise the current date is printed 

\begin{document}
\maketitle

\section{Johdanto}

Sulautetut laitteet ja IoT-ratkaisut eivät ole minulle tuttuja kehittäjänä. Olen käyttänyt monia sulautettuja laitteita kyllä arjessani, kuten esimerkiksi älykelloa ja älykästä pyykinpesukonetta. Laitteet keihtovat minua ja haluan oppia niistä lisää, sekä oppia myös kehittämän itse laitteita ja niiden ohjelmistoja. Päivätyössäni teen full stack -ohjelmointia, jossa työskennellään paljon datan käsittelyn parissa, luodaan rajapintoja muille sovelluksille ja integroidaan muiden järjestelmien tuottamaa tietoa omaan järjestelmäämme, joka yhdistää sen muihin tietoihin ja tarjoilee taas sitä eteenpäin. Olen myös suorittanut ammattikorkeakoulututkinnon, jossa pääosa opinnoistani liittyi ohjelmistotuotantoon ja digitaalisiin palveluihin, joten koen että minulla on hyvät valmiudet oppia uusia asioita kehitystyön tiimoilta. Arduinon tiedän nimeltä ja konseptilta, mutta en ole sellaista vielä kertaakaan käsitellyt edes simulaattorissa, joten siinä on varmasti täysin uusi asia edessä. Uskon kuitenkin, että verkosta löytyy runsaasti materiaalia, jolla sen kanssa pääsee nopeasti alkuun ja "hello world" tuntuu ainakin itselleni olevan se tärkein asia, eli että on olemassa toimiva ympäristö ja joku alkupiste, jossa tietää tekemiensä muutosten vaikuttavan ohjelmistoon.

Oppimistavoitteenani minulla on oppia sulautetuista järjestelmistä ja siitä millaista kehitystyö on niiden parissa konkreettisesti, sekä oppia kehittämään asioita Arduinolle. Työkaveriltani olen kuullut, että sulautettujen järjestelmien parissa työ on haasteellista, mutta palkitsevaa. Suurimpia oppimishaasteita ovat varmasti täysin uuden asian parissa työskentely ja siihen tarvittavan ajan raivaaminen kalenterista - siksi tämä karsimiskurssi on koulun kannalta varmasti erinomainen työkalu opiskelijan resurssien ja motivaation selvittämiseksi ennen varsinaisia opintoja.

\section{Referaatti}

More text.

\section{Lähteet}

Applications of Internet of Things in Manafacturing. Yang C., Weiming S., Xianbin W., 2016. Luettu: 01.04.2023. Luettavissa: https://learn.cinetcampus.fi/pluginfile.php/9512/mod_assign/intro/Applications%20of%20IoT%20in%20Manufacturing.pdf

\end{document}
